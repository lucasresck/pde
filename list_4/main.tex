\documentclass{article}
\usepackage[T1]{fontenc}
\usepackage[utf8]{inputenc}
\usepackage[portuguese]{babel}

\title{Lista 4 \\
\large Equações Diferenciais Parciais}
\author{Lucas Emanuel Resck Domingues}
\date{Outubro de 2020}

\usepackage{natbib}
\usepackage{graphicx}
\usepackage{amsmath}
\usepackage{hyperref}
\usepackage{listings}
\usepackage{amssymb}

%Hyperlinks
\usepackage{hyperref}
\hypersetup{
    colorlinks=true,
    allcolors=,  % Nothing change colors
    urlcolor=blue  % URL changes color
}

\begin{document}

    \maketitle
    
    \section*{Exercícios teóricos}

        \begin{enumerate}
            \item            
                \begin{enumerate}
                    \item Essa função é especial: ela é seno.
                        Sendo assim, podemos escolher os coeficientes
                        $a_n$ e $b_n$ tais que encontremos o seno de $5x$.
                        Da expressão da série de Fourier, basta que escolhamos
                        todos os coeficientes iguais a zero, com exceção de $b_5 = 1$.
                        Então:
                        
                        \begin{align*}
                            a(x) &= \dfrac{a_0}{2} + \sum_{m=1}^{\infty} \left(a_m \cos \left(\dfrac{m \pi x}{L}\right) + b_m \sin \left(\dfrac{m \pi x}{L}\right)\right) \\
                            &= \sin \left(\dfrac{5 \pi x}{L}\right) \\
                            &= \sin(5x)
                        \end{align*}

                    \item Calculemos os coeficientes:
                    
                        $$a_0 = \dfrac{1}{\pi} \int_{-\pi}^{\pi} \sin(5x + \alpha) dx = 0$$

                        Se $n \ne 0$:
                        
                        \begin{align*}
                            a_n &= \dfrac{1}{\pi} \int_{-\pi}^{\pi} \sin(5x + \alpha) \cos(nx) dx \\
                            &= \begin{cases}
                                -\dfrac{2n\sin(\alpha)\sin(\pi n)}{\pi(n^2-25)}, & n \ne 5\\
                                \sin(\alpha), & n = 5
                            \end{cases} \\
                            &= \begin{cases}
                                0, & n \ne 5\\
                                \sin(\alpha), & n = 5
                            \end{cases}
                        \end{align*}

                        Agora para $b_n$:

                        \begin{align*}
                            b_n &= \dfrac{1}{\pi} \int_{-\pi}^{\pi} \sin(5x + \alpha) \sin(nx) dx \\
                            &= \begin{cases}
                                -\dfrac{10\cos(\alpha)\sin(\pi n)}{\pi(n^2-25)}, & n \ne 5\\
                                \cos(\alpha), & n = 5
                            \end{cases} \\
                            &= \begin{cases}
                                0, & n \ne 5\\
                                \cos(\alpha), & n = 5
                            \end{cases}
                        \end{align*}

                        Então:
                        
                        \begin{align*}
                            b(x) &= \dfrac{a_0}{2} + \sum_{m=1}^{\infty} \left(a_m \cos \left(\dfrac{m \pi x}{L}\right) + b_m \sin \left(\dfrac{m \pi x}{L}\right)\right) \\
                            &= a_5 \cos(5x) + b_5 \sin(5x) \\
                            &= \sin(\alpha) \cos(5x) + \cos(\alpha) \sin(5x)
                        \end{align*}

                    \item
                        $$a_0 = \dfrac{1}{\pi} \int_{-\pi}^{\pi}c(x)dx = \dfrac{1}{\pi} \int_{0}^{\pi} 1 dx = 1$$

                        $$a_n = \dfrac{1}{\pi} \int_{0}^{\pi} \cos(nx) dx = 0$$

                        $$b_n = \dfrac{1}{\pi} \int_{0}^{\pi} \sin(nx) dx
                        = \dfrac{-\cos(n\pi)+1}{n\pi} = \begin{cases}
                            0, &n \text{ par} \\
                            \dfrac{2}{n\pi}, &n \text{ ímpar}
                        \end{cases}$$

                        $$c(x) = \dfrac{1}{2} + \sum_{n=1}^\infty \dfrac{2\sin(nx)}{n\pi}
                        = \dfrac{1}{2} + \dfrac{2}{\pi}\sum_{m=1}^\infty \dfrac{\sin((2m+1)x)}{(2m+1)}$$

                    \item
                        $$a_0 = \int_{-1}^{1}x dx = 0$$
                        
                        $$a_n = \int_{-1}^{1} x\cos(n\pi x) dx = 0$$
                        afinal o integrando é uma função ímpar.

                        $$b_n = \int_{-1}^{1} x \sin(n\pi x) dx = \dfrac{2 \sin(\pi n) - 2 \pi n \cos(\pi n)}{\pi^2n^2}$$

                        $$c(x) = \dfrac{2}{\pi^2} \sum_{n=1}^{\infty} \dfrac{\sin(\pi n) - \pi n \cos(\pi n)}{n^2} \sin(n\pi x)$$
                \end{enumerate}

            \item Veremos que a equação de diferenças parcial pode ser formulada
                como uma equação diferencial matricial. Vamos resolvê-la e verificar 
                que a base de Fourier nos ajuda a encontrar os coeficientes da solução
                da equação diferencial matricial.

                Temos um polígono com $n$ lados e $n$ vértices
                Sejam os vértices $x_1, \cdots, x_n$.
                Veja que a matriz de adjacência é dada:

                $$A = \begin{bmatrix}
                    0 & 1 & 0 & 0 & \cdots & 0 & 1 \\
                    1 & 0 & 1 & 0 &\cdots & 0 & 0 \\
                    0 & 1 & 0 & 1 &\cdots & 0 & 0 \\
                    \vdots & \vdots & \vdots & \vdots & \ddots & \vdots & \vdots \\
                    1 & 0 & 0 & 0 & \cdots & 1 & 0 \\
                \end{bmatrix}$$

                Então a matriz do laplaciano $L$ é dada por

                $$L = D - A = \begin{bmatrix}
                    2 & -1 & 0 & 0 & \cdots & 0 & -1 \\
                    -1 & 2 & -1 & 0 &\cdots & 0 & 0 \\
                    0 & -1 & 2 & -1 &\cdots & 0 & 0 \\
                    \vdots & \vdots & \vdots & \vdots & \ddots & \vdots & \vdots \\
                    -1 & 0 & 0 & 0 & \cdots & -1 & 2 \\
                \end{bmatrix}$$

                sendo $D$ a matriz diagonal com os graus de cada vértice.

                Nosso polígono regular tem $n$ lados. Considere o vetor $$w(t) = (u(x_1, t), u(x_2, t), \cdots, u(x_n, t))$$

                Ele descreve a temperatura em todos os vértices do polígono no tempo $t$.
                Porém, sabemos que

                $$w'(t) = -L w(t)$$

                (a simples multiplicação de $-L$ por $w$ resulta na equação de diferenças parcial do enunciado,
                porém para todos os valores de $x$ de uma vez só).
                Sendo assim, temos uma equação diferencial matricial.
                Observe que $L$ é uma matriz de convolução.
                
                Vamos considerar agora a base de Fourier $\{f_1, \cdots, f_n\}$, sendo

                $$f_i = \dfrac{1}{\sqrt{n}} \left((w^i)^0, (w^i)^1, \cdots, (w^i)^{n-1}\right)$$

                Nós sabemos que a base de Fourier é um conjunto ortonormal de cardinalidade $n$, então é uma base para o espaço
                $\mathbb{C}^n$. Ora, então existem $c_1(t), \cdots, c_n(t)$ tais que

                $$w(t) = \sum_{i=1}^n c_i(t)f_i$$

                Segue que

                $$w'(t) = \sum_{i=1}^n c_i'(t)f_i$$

                Porém, também sabemos que a base de Fourier diagonaliza matrizes de convolução. Portanto:

                $$\sum_{i=1}^n \dfrac{\mathrm{d}c_i}{\mathrm{d}t}(t)f_i = -L\sum_{i=1}^nc_i(t)f_i = -\sum_{i-1}^n c_i(t)\lambda_if_i$$

                sendo $\lambda_i$ autovalor associado ao autovetor $f_i$.
                O fato dos autovetores serem ortonormais nos leva a

                $$\dfrac{\mathrm{d}c_i}{\mathrm{d}t}(t) = -c_i(t)\lambda_i$$

                cuja solução conhecemos:

                $$c_i(t) = c_i(0) e^{-\lambda_i t}$$

                Para obtermos $c_i(0)$, fazemos

                $$\left<u(., 0), f_i\right> = \left<w(0), f_i\right> = \left<\sum_{i=1}^n c_i(0)f_i, f_i\right> = c_i(0)$$

                Para obtermos $\lambda_i$, já sabemos que basta $\lambda_i = \left<h, f_i\right>$,
                sendo $h$ o vetor que gera a matriz de convolução. Por exemplo, no nosso caso,

                $$h = (2, -1, 0, 0, \cdots, 0, -1)$$

                Resumindo: obtemos $\lambda_i$, $c_i(0)$, $c_i(t)$ e, finalmente, $w(t)$.
        \end{enumerate}

    \section*{Exercícios computacionais}

        \begin{enumerate}
            \item O programa para o Afinador de Fourier
                foi escrito em \textit{Matlab} e pode ser conferido
                no Apêndice \ref{appendix:tuner}.

                O programa toma como entrada o diretório do arquivo,
                lê o arquivo, calcula a transformada de Fourier do vetor, extrai
                a frequência com maior amplitude e calcula a frequência real
                com maior amplitude.

                Para o áudio \textit{CordaViolao2.wav}, o programa detectou
                predominância da frequência 331 Hz, que representa uma nota E,
                provavelmente a corda 1 do violão.
        \end{enumerate}

    \newpage

    \appendix

        \section{Afinador de Fourier}
            \label{appendix:tuner}

            \begin{lstlisting}[language=Matlab]
function freq = fourier_tuner(filepath)
    [y, Fs] = audioread(filepath);
    yhat = fft(y);
    % This is because of symmetry of Fourier coefficients
    max_freq = ceil(length(yhat)/2);
    [~, cicles] = max(abs(yhat(1:max_freq)));
    freq = cicles/(length(y)/Fs);
end
            \end{lstlisting}

\end{document}