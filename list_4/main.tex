\documentclass{article}
\usepackage[T1]{fontenc}
\usepackage[utf8]{inputenc}
\usepackage[portuguese]{babel}

\title{Lista 4 \\
\large Equações Diferenciais Parciais}
\author{Lucas Emanuel Resck Domingues}
\date{Outubro de 2020}

\usepackage{natbib}
\usepackage{graphicx}
\usepackage{amsmath}
\usepackage{hyperref}
\usepackage{listings}

%Hyperlinks
\usepackage{hyperref}
\hypersetup{
    colorlinks=true,
    allcolors=,  % Nothing change colors
    urlcolor=blue  % URL changes color
}

\begin{document}

    \maketitle
    
    \section*{Exercícios teóricos}

        \begin{enumerate}
            \item            
                \begin{enumerate}
                    \item Essa função é especial: ela é seno.
                        Sendo assim, podemos escolher os coeficientes
                        $a_n$ e $b_n$ tais que encontremos o seno de $5x$.
                        Da expressão da série de Fourier, basta que escolhamos
                        todos os coeficientes iguais a zero, com exceção de $b_5 = 1$.
                        Então:
                        
                        \begin{align*}
                            a(x) &= \dfrac{a_0}{2} + \sum_{m=1}^{\infty} \left(a_m \cos \left(\dfrac{m \pi x}{L}\right) + b_m \sin \left(\dfrac{m \pi x}{L}\right)\right) \\
                            &= \sin \left(\dfrac{5 \pi x}{L}\right) \\
                            &= \sin(5x)
                        \end{align*}

                    \item Calculemos os coeficientes:
                    
                        $$a_0 = \dfrac{1}{\pi} \int_{-\pi}^{\pi} \sin(5x + \alpha) dx = 0$$

                        Se $n \ne 0$:
                        
                        \begin{align*}
                            a_n &= \dfrac{1}{\pi} \int_{-\pi}^{\pi} \sin(5x + \alpha) \cos(nx) dx \\
                            &= \begin{cases}
                                -\dfrac{2n\sin(\alpha)\sin(\pi n)}{\pi(n^2-25)}, & n \ne 5\\
                                \sin(\alpha), & n = 5
                            \end{cases} \\
                            &= \begin{cases}
                                0, & n \ne 5\\
                                \sin(\alpha), & n = 5
                            \end{cases}
                        \end{align*}

                        Agora para $b_n$:

                        \begin{align*}
                            b_n &= \dfrac{1}{\pi} \int_{-\pi}^{\pi} \sin(5x + \alpha) \sin(nx) dx \\
                            &= \begin{cases}
                                -\dfrac{10\cos(\alpha)\sin(\pi n)}{\pi(n^2-25)}, & n \ne 5\\
                                \cos(\alpha), & n = 5
                            \end{cases} \\
                            &= \begin{cases}
                                0, & n \ne 5\\
                                \cos(\alpha), & n = 5
                            \end{cases}
                        \end{align*}

                        Então:
                        
                        \begin{align*}
                            b(x) &= \dfrac{a_0}{2} + \sum_{m=1}^{\infty} \left(a_m \cos \left(\dfrac{m \pi x}{L}\right) + b_m \sin \left(\dfrac{m \pi x}{L}\right)\right) \\
                            &= a_5 \cos(5x) + b_5 \sin(5x) \\
                            &= \sin(\alpha) \cos(5x) + \cos(\alpha) \sin(5x)
                        \end{align*}

                    \item
                        $$a_0 = \dfrac{1}{\pi} \int_{-\pi}^{\pi}c(x)dx = \dfrac{1}{\pi} \int_{0}^{\pi} 1 dx = 1$$

                        $$a_n = \dfrac{1}{\pi} \int_{0}^{\pi} \cos(nx) dx = 0$$

                        $$b_n = \dfrac{1}{\pi} \int_{0}^{\pi} \sin(nx) dx
                        = \dfrac{-\cos(n\pi)+1}{n\pi} = \begin{cases}
                            0, &n \text{ par} \\
                            \dfrac{2}{n\pi}, &n \text{ ímpar}
                        \end{cases}$$

                        $$c(x) = \dfrac{1}{2} + \sum_{n=1}^\infty \dfrac{2\sin(nx)}{n\pi}
                        = \dfrac{1}{2} + \dfrac{2}{\pi}\sum_{m=1}^\infty \dfrac{\sin((2m+1)x)}{(2m+1)}$$

                    \item
                        $$a_0 = \int_{-1}^{1}x dx = 0$$
                        
                        $$a_n = \int_{-1}^{1} x\cos(n\pi x) dx = 0$$
                        afinal o integrando é uma função ímpar.

                        $$b_n = \int_{-1}^{1} x \sin(n\pi x) dx = \dfrac{2 \sin(\pi n) - 2 \pi n \cos(\pi n)}{\pi^2n^2}$$

                        $$c(x) = \dfrac{2}{\pi^2} \sum_{n=1}^{\infty} \dfrac{\sin(\pi n) - \pi n \cos(\pi n)}{n^2} \sin(n\pi x)$$
                \end{enumerate}
        \end{enumerate}

    \newpage

    \appendix

\end{document}